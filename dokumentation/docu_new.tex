\documentclass[a4paper, 10pt]{scrreprt}

\pagenumbering{roman}
\setcounter{secnumdepth}{3}
\setcounter{tocdepth}{2} 

\usepackage[german]{babel}
\usepackage[utf8]{inputenc}
\usepackage[T1]{fontenc}
%\usepackage{fancyhdr}
\usepackage{geometry}
\usepackage{lmodern}
\usepackage{verbatim}
\usepackage[pdfborder={0 0 0}]{hyperref}

\usepackage{amsmath}
\usepackage{blindtext}


\usepackage{fancybox}
\usepackage{makeidx}% \makeindex
\usepackage{lastpage}
\usepackage{natbib}

%\linespread{1.5}
%\geometry{footskip=40pt}
\setcounter{page}{2}
\linespread{1.5}


\begin{document}

\begin{titlepage}
%\includegraphics[scale=0.3]{Z1.png} \hfill
%\includegraphics[scale=0.1]{2.png}\\[1.8cm]
    \begin{center}
    \LARGE \textbf{VirtualDesktop} \\
    \vspace{2.5cm}
    \large\textbf{Projektarbeit}\\
    \vspace{2.5cm}
    \normalsize
    Hochschule RheinMain Wiesbaden \\
    \vspace{2cm}
    \large \textbf{Fortgeschrittene Themengebiete der Informatik\\ Cloud Computing\\}
    \vspace{1cm}
    \normalsize
    Abgabedatum: 30. Januar 2019\\
    \vspace{2.7cm}
    \end{center}
 \normalsize{
    \begin{tabular}{ll}
    	Gruppe: & \\
    	Robin Bergfeld & \\
    	Abiram Pakeerathan & \\
    	Simon Rininsland & \\[0.5cm]
    	Dozent: &\\
        Prof. Dr. Philipp Schaible & \\
    \end{tabular}
    }
\end{titlepage}



\clearpage
\tableofcontents
\clearpage

\pagenumbering{arabic}

%zitieren:
% \cite{Joa10:1}

\chapter{Einführung}
Mit was, wer wo...

\section{Projektanforderungen}
- Welche Anforderungen vom Modul\\
- Fokus des Projekts

\section{Produktvision}
- Was mit Bildern (kurz, vielleicht nur ein Bild)\\
- Funktionen erläutern (MultiWindow, Benutzerverwaltung inkl Rechteverwaltung, Stream, Up-Download, Show)\\
- Ziel des Produkts (müssen sich mit Anforderungen des Moduls decken)\\


\chapter{Architektur}
einleitung für archtektur.

\section{unterüberschriften? mehrere}
- Festlegung auf AWS\\
- Erklären was Hardware Architektur einschließt\\
- Was benötigen wir an Services (Speicher, DB, Auth, Scale)\\
- Welche Systeme und Services nutzen wir iin AWS mit kurzer Erklärung (s3, dyndb, cognito, beanstalk, lambda)\\
- Architektur zeigen und einzeln mit langer Erklärung wie Services zusammenhängen und miteinander arbeiten. Inkl aller Subservices die EBS anlegt.(LoadBalancer, Security Groups, S3 Bucket für EB Code, Cloudwatch)\\
- Welche services davon müssen warum skaliert werden

\section{architeturdiagramm?}

\chapter{Umsetzung Software?}
Einleitung Anwendung

\section{NodeJS und Express}
- Was ist das. Wofür brauchen wir das\\
- Erklären wieso wir das nutzen.\\
- Erklären Zusammenhang JSON zu Backend datentransform und weswegen NodeJS deswegen gut.

\section{Frontend JS / jQuery}
- Was sind das für Technologien. Wofür brauchen wir das\\
- Erklären wieso wir das nutzen. \\
- Erklären Zusammenhang JSON zu Backend Datentransfer und weswegen NodeJS deswegen gut.


\chapter{Interessante Themen}
Einleitung dazu: Weil zu viel gemacht - nur die interessante Teile näher erklärt.

\section{Entwicklungsumgebung lokal / Env- Variablen in NodeJS}
- Was ist das Problem.\\ 
- Wie haben wir es gelöst\\
-  z.B. www Cognito

\section{DynamoDB}
- Was ist das Problem. Nicht mehrere Sachen löschen nicht möglich. 

\section{Promise / Dispatcher}
- Was ist das Problem. Wofür brauchen wir das. Wo war das problem.\\
- Erklären wieso wieso wir das bauen.\\
- Erklären was wir gebaut haben und wie.\\
- kann easy erweitert werden

\section{Controller}
- Warum brauchen wir das\\
- kann easy erweitertert werden\\
- Wie haben wir es gelöst

\section{Streaming Download / Upload}
- Was ist das Problem. Wofür brauchen wir das. Wo war das problem.\\
- Erklären wieso wieso wir das bauen.\\
- Erklären was wir gebaut haben und wie.

\section{Lambda - Thumbnailgenerator}
- Was macht das\\
- Wie geht das

\chapter{Kosten - Aufwände}

\section{Laufkosten}
- Aufführung der genutzten Services mit Bezug auf Kosten\\
- Hochrechnung pro Monat und pro Nutzer (oder pro 100 Nutzer) \\
- Grafik erstellen







\clearpage
\bibliography{documentation}
%\bibliographystyle{apalike}

\end{document}